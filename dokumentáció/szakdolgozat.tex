\documentclass[a4paper,12pt]{report}
\usepackage[T1]{fontenc}
\usepackage[utf8]{inputenc}
\def\magyarOptions{defaults=hu-min}
\usepackage[magyar]{babel}
\usepackage{amsthm, amssymb,amsmath,hyperref}
\usepackage{enumerate, graphicx, xcolor}

\usepackage{chngcntr}
\counterwithout{figure}{chapter}
\counterwithout{figure}{section}
\counterwithout{figure}{subsection}
%\usepackage{pgf,tikz,float}
%\usepackage{tikzlings}
%\usepackage{tikzducks}
%\usetikzlibrary{arrows}
\usepackage[nobysame]{amsrefs}
%\usepackage{amsmath}

\usepackage{geometry}
   \geometry{
   a4paper,
   total={160mm,247mm},
   left=25mm,
   top=25mm,
}


\newtheorem{theo}{tétel}[section]
\newtheorem{defin}[theo]{definíció}
\newtheorem{lemma}[theo]{lemma}
\newtheorem{all}[theo]{állítás}
\newtheorem{kov}[theo]{következmény}

\theoremstyle{definition}
\newtheorem{definition}[theo]{definíció}

\theoremstyle{remark}
\newtheorem{megj}[theo]{megjegyzés}




\date{today}

\linespread{1.3}



\begin{document}
\thispagestyle{empty}

\begin{center}
   {\Large A dolgozat formai követelményei}   
\end{center}

\vspace{1 cm}

{\bf Ajánlott oldalszám:}
\begin{itemize}
\item BSc szakdolgozat: 20-25 oldal, 
\item MSc diplomamunka: 35-50 oldal
\item Tanárszakos szakdolgozat: 50000-80000 karakter szóközökkel (lásd. TKK szabályzat)
\end{itemize}

\vspace{1 cm}

{\bf Másfeles sorköz, sorkizárt, 12-es betű méret (általában Computer Modern), margók: jobb, bal, lent és fent 2,5 cm} (ahogy ez a template van beállítva, az jó).

\vspace{1 cm}

{\bf Kötelező elemei a dolgozatnak}
\begin{itemize}
   \item Címlap
   \item Tartalmi összefoglaló (kivonat)
   \item Tartalomjegyzék
   \item \textcolor{red}{Érdemi rész}
(Szakterület specifikus fejezeteket tartalmaz, kérjük, konzultáljon a témavezetővel, hogy a példában közzétett kísérletes tudományterületeken használt fejezetekből az Ön dolgozatában melyikre van szükség!)
\item Irodalomjegyzék
\item Nyilatkozat
\end{itemize}
\newpage



\pagenumbering{roman}

%Elso  oldal 
\thispagestyle{empty}

\begin{center}
\vspace*{0.2cm} {\Large\bf Szegedi Tudományegyetem}
\vspace{0.3cm}

{\Large\bf Természettudományi és Informatikai Kar}
\vspace{0.3cm}

{\Large\bf XXXXXXX Intézet, Számítástudomány Alapjai Tanszék}
\vspace{3cm}



{\Large SZAKDOLGOZAT}

\vspace*{1.5cm}

{\LARGE\bf A szakdolgozat címe}

opcionálisan, formázatlanul
A szakdolgozat angol címe 



\vspace*{4cm}

{\large
\begin{tabular}{c@{\hspace{2cm}}c}
\emph{Készítette:}     &\emph{Témavezető:}\\
\bf{Kozma Kristóf}  &\bf{Dr. Iván Szabolcs}\\
Programtervező Informatikus BSc hallgató    & egyetemi docens\\
&
\end{tabular}
}

\vspace*{1,5cm}

{\Large Szeged\\ \vspace{2mm} 20XX}
\end{center}

%masodik oldal osszefogalalo
\begin{abstract}
A dolgozat tartalmának rövid (max. 1 oldal) összefoglalása. A következő részekből áll: rövid irodalmi összefoglaló, a dolgozat elkészítéséhez használt módszerek, eredmények, konklúzió

{\bf Kulcsszavak:} a dolgozat tartalmára specifikusan jellemző 4-6 szó, egymástól vesszővel elválasztva
\end{abstract}



\newpage


\pagebreak

\tableofcontents
\pagebreak
%\listoffigures
%\pagebreak



\chapter{Bevezetés}
\pagenumbering{arabic}

(a fejezet új oldalon kezdve)

Tartalmazza a problémafelvetést, a témaválasztás indoklását és a munka célját.

A célkitűzések részletes megfogalmazása külön fejezetben az irodalmi áttekintés után legyen.



\chapter*{Motiváció}

\chapter*{Piackutatás}

\chapter*{Funkcionális specifikáció}

\chapter*{Használt technológiák}

\chapter*{Modulok}

\chapter*{Adatmodell}

\chapter*{A rendszer magasszintű folyamatai, működése}

\chapter*{Fontosabb kódrészletek}

\chapter*{Tesztelés}

\chapter*{Tapsztalatok, továbbfejlesztési lehetőségek}


\chapter{Összefoglalás}

(új oldalon kezdve)

A dolgozat eredményeinek összefoglalása, következtetések levonása.

Az összefoglalásban egyértelműen jelezve legyen a hallgató saját szerepe/eredményei.


\newpage
{\Huge \bf Köszönetnyilvánítás}

\addcontentsline{toc}{chapter}{Köszönetnyilvánítás}

\vspace{2 cm}

(nem kötelező elem), (új oldalon kezdve) 

Ebben a fejezetben lehet köszönetet mondani mindazoknak, akik segítették a dolgozat elkészülését. Itt lehet megemlíteni továbbá a munkát támogató pályázatokat, ösztöndíjakat, stb.

\newpage
{\Huge \bf Nyilatkozat}

\addcontentsline{toc}{chapter}{Nyilatkozat}

\vspace{2 cm}

Alulírott, Kozma Kristóf, Programtervező Informatikus BSc szakos hallgató, kijelentem, hogy a szakdolgozatban ismertetettek saját munkám eredményei, és minden felhasznált, nem saját munkából származó eredmény esetén hivatkozással jelöltem annak forrását. 


\begin{flushleft}
\vspace*{1cm}
Szeged, \today
\end{flushleft}

\begin{flushright}
   \vspace*{1cm}
   \makebox[7cm]{\rule{6cm}{.4pt}}\\
   \makebox[7cm]{\emph{Kozma Kristóf}}
\end{flushright}

\end{document}



